%Corps du document :
%\setlength{\parindent}{1cm}    

\section{Introduction}

Le but du projet ``Système d'Information Urbanisé et SOA'' est de mettre en application la démarche et
les méthodes de conception de systèmes d'information vues en cours dans le cadre d'une architecture répartie.
Il s'agira de fournir une solution logicielle pour ``la gestion des contacts commerciaux'' d'une banque,
en apportant une aide à ses agents commerciaux pour :

\begin{itemize}
\item Identifier et définir les contacts qu'ils doivent avoir avec leurs clients,
\item Permettre au chef d'agence de répartir ces contacts entre ses collaborateurs,
\item Prendre les rendez-vous et tenir leurs agendas,
\item Préparer ces rendez-vous et les projets de proposition en fonction des clients,
\item Construire les entretiens lors des rendez-vous et déclarer les résultats obtenus,
\item Suivre la réalisation des contacts programmés.
\end{itemize}

Ce projet se déroulera en 3 phases :

\begin{itemize}
\item La conception d'ensemble de l'architecture applicative,
\item La conception détaillée des applications,
\item La répartition des composants sur l'architecture n-tiers.
\end{itemize}

Nous utiliserons l'outil de suivi Red Mine pour la conduite de
projet, la répartition et l'organisation des tâches, et
l'établissement d'un planning prévisionnel.

\section{Répartition des tâches}

\begin{itemize}
\item Conception d'ensemble du système

\begin{itemize}
\item Analyse des modèles conceptuels proposés (assigné à \textbf{Xavier Sauvagnat})
\item Etablissement des diagrammes d'activités (assigné à \textbf{Matthieu Coquet})
\item Etablissement des diagrammes de séquences systèmes (assigné à \textbf{Alexandre Lefoulon})
\item Identifier et définir les blocs applicatifs

\begin{itemize}
\item Etablissement du modèle de données découpés en blocs (assigné à \textbf{Alexandre Lefoulon})
\end{itemize}

\item Identifier les cycles de vie des objets métiers

\begin{itemize}
\item Etablissement du diagramme d'état d'objet métier, objet Contact UNIQUEMENT (assigné à \textbf{Jan Keromnes})
\end{itemize}

\item Détermination des flux de l'architecture

\begin{itemize}
\item Etablissement d'un diagramme de séquence (assigné à \textbf{Quentin Calvez})
\item Etablissement d'un diagramme de collaboration(assigné à \textbf{Thaddee Tyl})
\end{itemize}

\item Choix de l'environnement technique (assigné à \textbf{Matthieu Coquet})

\end{itemize}

\item Conception détaillée

\begin{itemize}
\item Interface utilisateur

\begin{itemize}
\item Diagrammes EdF (assigné à \textbf{Jan Keromnes})
\item Description des fenêtres (assigné à \textbf{Quentin Calvez})
\item Services IHM (assigné à \textbf{Jan Keromnes})
\end{itemize}

\item Couche noyau

\begin{itemize}
\item Spécification des SM (assigné à \textbf{Xavier Sauvagnat})
\item Spécification des SOM(assigné à \textbf{Alexandre Lefoulon})
\end{itemize}

\end{itemize}

\item Architecture technique et répartition du SI

\begin{itemize}
\item Choix architecture Centralisée/Réparties (assigné à \textbf{Quentin Calvez})
\item Choix de la répartition des composants applicatifs (assigné à \textbf{Thaddee Tyl})
\item Détermination des principaux flux au sein de l'application, CRUD (assigné à \textbf{Jan Keromnes})
\end{itemize}

\end{itemize}

\section{Evaluation des charges}



\section{Diagramme de Gantt}


