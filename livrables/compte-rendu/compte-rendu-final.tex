%-------------------------------------------
% En-tête type de document pour le projet PLD
% Il suffit de remplir le input ligne 45
%-------------------------------------------

\documentclass[a4paper]{article}

\usepackage[utf8]{inputenc}   
\usepackage[top=2cm, bottom=2cm, left=2cm, right=2cm]{geometry}
\usepackage{ucs}
% Reconnaitre les caratères accentués dans le source.
\usepackage[T1]{fontenc} 
\usepackage{lmodern}
\usepackage[francais]{babel}
% Insertion d'images
\usepackage{graphicx}
% Utilisation du symbole EURO
\usepackage{eurosym}

\begin{document}

%------------------------------------- Page de titre
\begin{titlepage}
~ 
\vfill
	\begin{center}
		\begin{Huge}
		Projet Système d'Information Urbanisé \& SOA : Compte Rendu\\
		\end{Huge} 
\vfill
		\textbf{Hexanome 4111 :} 
		\\Quentin \bsc{Calvez}, Matthieu \bsc{Coquet}, 
		\\Jan \bsc{Keromnes}, Alexandre \bsc{Lefoulon}, 
		\\Xavier \bsc{Sauvagnat, }Thaddée \bsc{Tyl},
		\\Tuuli \bsc{Tyrväinen}
\vfill		
		\begin{Large}
		Février 2012
		\end{Large}
\vfill
	\begin{tabular}{|c|c|c|c|c|}
 	 \hline
   Destinataire & Version & Etat & Dernière révision & Equipe \\
   \hline
   Client & 1 & Validé & \today & H4111 \\
   \hline
	\end{tabular}
	\end{center}
\vfill
\end{titlepage}
%----------------------------------------------------
%--------------------------------- Table des matières
\newpage
\tableofcontents
\newpage
%----------------------------------------------------

%------------------- Insertion du contenu du document

%Corps du document :
%\setlength{\parindent}{1cm}    

\section{Introduction}


%----------------------------------------------------

\end{document}
