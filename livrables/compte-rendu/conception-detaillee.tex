%Corps du document :
%\setlength{\parindent}{1cm}    

\section{Conception détaillée}

Attachons nous maintenant à la conception détaillée de notre application. Il s'agit
d'identifier et de spécifier les composants nécessaires pour automatiser tout ou
partie des outils à utiliser dans le cadre des cas d'utilisation identifiés.

Commençant par spécifier l'enchaînement des fenêtres grâce à un diagramme.

\subsection{Diagramme d'enchaînement des fenêtres}

\begin {center}
\includegraphics[width=\textwidth]{diagramme-edf.png}
\end {center}

\paragraph{Description}

L'Interface Homme-Machine (IHM) sera composée de trois onglets
principaux :

\begin{itemize}
\item \textbf{L'onglet Contacts} : Il présente la liste des Contacts prévus et affectés.
\item \textbf{L'onglet Agenda} : Il permet de consulter la liste des RDV pris.
\item \textbf{L'onglet Clients} : Cet onglet permet d'accéder aux dossiers des clients de la banque.
\end{itemize}

On observera également le volet \textbf{Détails du Contact}. Il permet d'effectuer toutes les
opérations nécessaires sur un Contact (une liste de ces services métiers se trouve plus loin
dans ce compte rendu).

\subsection{Dessin des fenêtres de l'IHM}

Voici une première représentation graphique de l'IHM que nous contruirons :

TODO @quentez

\subsection{Services Métiers invoqués par l'IHM}

Cette IHM utilise différents services métiers. En voici la liste :

\subsubsection{Services Métier Client}

\paragraph{Recherche de Clients}

\begin{itemize}
\item getAllClients()
\end{itemize}

\paragraph{Dossier Client}

\begin{itemize}
\item afficherInfos(idClient)
\item updateClient(client)
\end{itemize}

\subsubsection{Services Métier Agenda}

\begin{itemize}
\item getAgenda(type)
\item getSemaineParAgent(semaine,idAgent)
\item getJourAgence(date)
\item getPlageHoraire(idAgent,date)
\item planifierPlageAgenda(idAgent,typeActivite,date)
\item terminerPlanification()
\end{itemize}

\subsubsection{Services Métier Contact}

\paragraph{Recherche de Contacts}

\begin{itemize}
\item getAllContacts(idAgence)
\end{itemize}

\paragraph{Affectation des Contacts}

\begin{itemize}
\item getAllAgents(idAgence)
\item getContactsAgent(idAgent)
\item affecterContactAgent(idAgent,idContact)
\item transférerContact(idContact,idAgent)
\end{itemize}

\paragraph{Gestion d'un Contact}

\begin{itemize}
\item updateContact(contact)
\item annulerContact(idContact,raison)
\item changerEtatContact(date,idContact,etat)
\item confirmerContact(lettre)
\item réaliserContact(idContact)
\item regrouperContacts(liste<Contact>)
\item créerContactSpontané(date,idClient)
\end{itemize}

\paragraph{Préparation de l'entretien}

\begin{itemize}
\item consulterCatalogue(idSegment,idProduit)
\item préparerContact(idContact)
\item consulterPréparation(idContact)
\end{itemize}

\paragraph{Rapport d'activité commerciale}

\begin{itemize}
\item soumettreProposition(idContact,idOffre)
\item soumettreRapport{iContact,rapport}
\end{itemize}

\subsection{Spécification des Services Métier}

On ne détaillera que la liste des Services Métier invoqué par l'IHM dossier Client:

SM1: afficherInfo
\begin{itemize}
	\item SOM invoqués : SOMC1 : Consultation informations client
	\item Entrées : idClient
	\item Sorties : NomClient, Adresse, Comptes,....
	\item Procédure :
	\begin{itemize}
		\item Rechercher les informations le client (personne,compte): SOMC5.
	\end{itemize}
\end{itemize}

SM2 : updateClient
\begin{itemize}
	\item SOM invoqués : SOMC2 : Mise à jour des informations client
	\item Entrées : NomClient/Adresse/Compte/... (uniquement ceux modifiées)
	\item Sorties : null
	\item Procédure :
	\begin{itemize}
		\item Mettre à jour les informations sur le client (personne,compte): SOMC4.
	\end{itemize}
\end{itemize}

\subsection{Spécification des Services Objets Métier}

On ne détaillera que les Services Objets Métier concernant le bloc Client:


OM Client SOMC1 : getInfo
\begin{itemize}
\item Entrées : idClient
\item Sorties : NomClient, Adresse, Comptes,...
\item Entités/Classes : Client, Personne, Adresse, Compte
\item Permet de consulter toutes les informations d'un client.
\end{itemize}

OM Client SOMC2 : updateInfo
\begin{itemize}
\item Entrées : idClient, info
\item Sorties : null
\item Entités/Classes : Client, Personne, Adresse, Compte
\item Permet de mettre à jour des informations sur un client.
\end{itemize}

OM Client SOMC3 : getAll
\begin{itemize}
\item Entrées : null
\item Sorties : liste Client: NomClient, Adresse,...
\item Entités/Classes : Client, Personne, Adresse, Compte
\item Permet de consulter la liste de tous les clients.
\end{itemize}

\subsection{Spécification des IHM}
Présentation de la vue de login
\includegraphics[width=\textwidth]{../../ihm/pngIHM/login.png}
Présentation de la vue Contact
Chaque contact est donc sélectionnable via la partie de gauche, pour chaque contact on a ensuite son détail dans la partie droite.
\includegraphics[width=\textwidth]{../../ihm/pngIHM/ContactEntretiens1.png}
En scrollant la partie droite (via l'onglet Entretien) on obtient l'état d'un entretien préparé ou déjà effectué.
\includegraphics[width=\textwidth]{../../ihm/pngIHM/ContactEntretiens2.png}
\includegraphics[width=\textwidth]{../../ihm\pngIHM/ChoixAgent.png}
\includegraphics[width=\textwidth]{../../ihm/pngIHM/Calendrier Hebdomadaire.png}
\includegraphics[width=\textwidth]{../../ihm/pngIHM/CalendrierQuotidien.png}
\includegraphics[width=\textwidth]{../../ihm/pngIHM/ListeClients.png}
\includegraphics[width=\textwidth]{../../ihm/pngIHM/ListeContacts.png}
\includegraphics[width=\textwidth]{../../ihm/pngIHM/NouveauContactSpontanne.png}


