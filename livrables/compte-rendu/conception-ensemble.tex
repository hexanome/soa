%Corps du document :
%\setlength{\parindent}{1cm}    

\section{Conception d'ensemble de l'architecture applicative}

\subsection{Diagrammes d'activité}

\begin {center}
\includegraphics[width=\textwidth]{../../diagrammeActivite/DACU1.png}
\includegraphics[width=\textwidth]{../../diagrammeActivite/DACU2.png}
\includegraphics[width=\textwidth]{../../diagrammeActivite/DACU3.png}
\includegraphics[width=\textwidth]{../../diagrammeActivite/DACU4.png}
\includegraphics[width=\textwidth]{../../diagrammeActivite/DACU5.png}
\includegraphics[width=\textwidth]{../../diagrammeActivite/DACU6.png}
\includegraphics[width=\textwidth]{../../diagrammeActivite/DACU7.png}
\includegraphics[width=\textwidth]{../../diagrammeActivite/DACU8.jpg}
\includegraphics[width=\textwidth]{../../diagrammeActivite/DACU9.jpg}
\includegraphics[width=\textwidth]{../../diagrammeActivite/DACU10.png}
\end {center}

\subsection{Blocs applicatifs}
Les modèles conceptuels de données clients et produits ainsi que commercial nous étant fournis, nous avons simplement effectuer un découpage en blocs applicatifs qui permettent d'avoir des communications entre les différents blocs équilibrés.
Une modification a été apportée sur le modèle commercial, afin de pouvoir rendre réalisable le CU6, plus particulièrement la possibilité d'inscrire dans l'agenda une rencontre spontanée, qui donne lieu à un contact réalisé sans association de contact prévu, d'où l'ajout d'une relation entre poste fonctionnel (agent) et contact réalisé.
\begin {center}
\includegraphics[width=\textwidth]{Decoupage MCD 1.png}
\includegraphics[width=\textwidth]{Decoupage MCD 2.png}
\end {center}

\subsection{Cycles de vie des objets métiers}

Voici le diagramme d'état de l'objet ``Contact'' :

\begin {center}
\includegraphics[width=\textwidth]{diagramme-etat-objet-contact.png}
\end {center}
Comme prévu, il n'est pas possible de réaffecter un contact à un autre agent. En effet un agent sera toujours responsable du groupe de contacts qu'on lui a affectés. Néanmoins, les rendez-vous pris, eux, pourront être réaffectés à un autre agent que celui en charge du contact. Cela permettra entre autre de palier à l'absence d'un agent le jour du rendez-vous client.


\subsection{Détermination des flux de l'architecture}

\subsubsection{Diagramme de séquence}

Ci-dessous sont présentés les différents diagrammes de séquences :
\begin {center}
\includegraphics[width=\textwidth]{../../webSequenceDiagrameSources/cu1.png}
\includegraphics[width=\textwidth]{../../webSequenceDiagrameSources/cu2.png}
\includegraphics[width=\textwidth]{../../webSequenceDiagrameSources/cu3.png}
\includegraphics[width=\textwidth]{../../webSequenceDiagrameSources/cu4.png}
\includegraphics[width=\textwidth]{../../webSequenceDiagrameSources/cu5.png}
\includegraphics[width=\textwidth]{../../webSequenceDiagrameSources/cu6.png}
\includegraphics[width=\textwidth]{../../webSequenceDiagrameSources/cu7.png}
\includegraphics[width=\textwidth]{../../webSequenceDiagrameSources/cu8.png}
\includegraphics[width=\textwidth]{../../webSequenceDiagrameSources/cu9.png}
\includegraphics[width=\textwidth]{../../webSequenceDiagrameSources/cu10.png}
\end {center}

\subsubsection{Diagramme de collaboration}

Voici le diagramme de collaboration représentatif de la dynamique de
l'architecture:

\begin{figure}[H]
\centering
\includegraphics[scale=0.7,angle=90]{diagramme-collaboration.png}
\caption*{Diagramme de collaboration.}
\end{figure}

L'objectif du diagramme de collaboration est de vérifier le découpage en blocs applicatifs et si besoin de refaire l'étape d'analyse pour le découpage. Dans notre cas, il sert juste à confirmer notre découpage. Il nous sert également à appuyer le choix de l'architecture retenue, qui est discuter dans la partie suivante.


\subsection{Choix de l'environnement technique}
L'environnement technique sera une architecture C/S 3-tiers : l'architecture du système sera donc séparée en trois couches distinctes, à savoir :
\begin{itemize}
\item Couche de données : l'information est stockée ici sur la ou les bases de données. Ces informations seront récuperé par la couche applicative
\item Couche applicative : représente le coeur de l'architecture. A chaque demande de la couche de présentation, la couche applicative va effectuer un calcul, appliquer les règles de gestion et au besoin récuperer les informations de la base de données.
\item Couche de présentation : à chaque action de l'utilisateur, cette couche va d'une part transférer l'information créée par l'utilisateur à la couche applicative et d'autre part afficher l'information récupérée et traitée depuis la base de données par les deux couches précédentes.
\end{itemize}
Le détail de ces tiers sera présenté dans le chapitre 'Architecture technique', notamment la localisation et la répartition des données.
