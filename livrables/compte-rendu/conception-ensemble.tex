
%Corps du document :
%\setlength{\parindent}{1cm}    

\section{Conception d'ensemble de l'architecture applicative}


\subsection{Cycles de vie des objets métiers}

Voici le diagramme d'état de l'objet ``Contact'' :

\begin {center}
\includegraphics[width=\textwidth]{diagramme-etat-objet-contact.png}
\end {center}
Comme prévu, il n'est pas possible de réaffecter un contact à un autre agent. En effet un agent sera toujours responsable du groupe de contacts qu'on lui a affectés. Néanmoins, les rendez-vous pris, eux, pourront être réaffectés à un autre agent que celui en charge du contact. Cela permettra entre autre de palier à l'absence d'un agent le jour du rendez-vous client.

\subsection{Choix de l'environnement technique}
L'environnement technique sera une architecture C/S 3-tiers : l'architecture du système sera donc séparée en trois couches distinctes, à savoir :
\begin{itemize}
\item Couche de données : l'information est stockée ici sur la ou les bases de données. Ces informations seront récuperé par la couche applicative
\item Couche applicative : représente le coeur de l'architecture. A chaque demande de la couche de présentation, la couche applicative va effectuer un calcul, appliquer les règles de gestion et au besoin récuperer les informations de la base de données.
\item Couche de présentation : à chaque action de l'utilisateur, cette couche va d'une part transférer l'information créée par l'utilisateur à la couche applicative et d'autre part afficher l'information récupérée et traitée depuis la base de données par les deux couches précédentes.
\end{itemize}
Le détail de ces tiers sera présenté dans le chapitre 'Architecture technique', notamment la localisation et la répartition des données.